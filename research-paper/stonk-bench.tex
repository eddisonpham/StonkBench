\documentclass[sigconf]{acmart}
\usepackage{enumitem}
\AtBeginDocument{%
  \providecommand\BibTeX{{%
    Bib\TeX}}}

% \setcopyright{acmlicensed}
\copyrightyear{2025}
\acmYear{2025}
% \acmDOI{XXXXXXX.XXXXXXX}
% \acmConference[Conference acronym 'XX]{Make sure to enter the correct
%   conference title from your rights confirmation email}{June 03--05,
%   2018}{Woodstock, NY}
% \acmISBN{978-1-4503-XXXX-X/2018/06}



\begin{document}


\title{Stonk Bench: Unified Benchmark for Synthetic Data Generation for Financial Time Series}



\author{Uyen Lam Ho}
\authornote{Both authors contributed equally to this research.}
\email{uyenlam.ho@mail.utoronto.ca}
\author{Eddison Pham}
\authornotemark[1]
\email{eddison.pham@mail.utoronto.ca}
\affiliation{%
  \institution{University of Toronto}
  \city{Toronto}
  \state{Ontario}
  \country{Canda}
}



\renewcommand{\shortauthors}{Ho and Pham}

\begin{abstract}
  The ever emerging field of Synthetic Data Generation (SDG) has gain traction 
  within many financial domains, including Finacial Time Series (FTS) data. 
  However, there has been discorse and lack in universal agreement in what determines a better SDGFTS 
  (Synthetic Data Generation for Financial Time Series), which maybe hindering progress 
  in the field. 
  In this paper, we propose and contribute a comprehensive benchmark for SDGFTS, 
  covering various aspects such as data quality, privacy, and utility. 
  We evaluate several state-of-the-art SDG methods using our benchmark and 
  provide insights into their strengths and weaknesses. 
  Our first-of-its-kind benchmark aims to facilitate the development of more effective SDG 
  methods for FTS data, incorperating both classical statistical measures
  and utility benefits, and test over (insert number) to decicively determine what is the 
  best SDGFTS model right now and for the future.
\end{abstract}



\keywords{
    Synthetic Data Generation, 
    Financial Time Series,
    Benchmarking
}
\begin{teaserfigure}
  \centering
  \includegraphics[width=\textwidth,height=5.5cm,keepaspectratio,trim=0 150 0 150,clip]{tse-1981.png}
  \caption{Toronto Stock Exchange, Toronto, 1981}
  \Description{A panoramic view of the Toronto Stock Exchange building in 1981.}
  \label{fig:teaser}
\end{teaserfigure}


\maketitle

\section{Introduction}
  Synthetic Data Generation (SDG) has emerged as a crucial tool in financial technology, 
  particularly for Financial Time Series (FTS) data \cite{potluru24survey}. The ability to generate high-quality 
  synthetic financial data addresses several critical challenges in the field, 
  including data privacy concerns, limited data availability, and the need for diverse training 
  datasets in machine learning applications \cite{Stenger24}.

  Despite the growing importance of SDGFTS (Synthetic Data Generation for 
  Financial Time Series), there is a notable lack of standardization in evaluating 
  the quality and effectiveness of these generative models \cite{Stenger24}. Current evaluation methods 
  vary widely across studies, making it difficult to compare different approaches objectively 
  and determine their relative strengths and weaknesses.

  Our research addresses this gap by introducing a comprehensive benchmark framework that 
  encompasses:
  \begin{itemize}
      \item Statistical fidelity measures for comparing synthetic and real FTS
      \item Privacy preservation metrics to ensure sensitive financial information 
      remains protected
      \item Utility metrics that assess the practical value of synthetic data in 
      downstream tasks
      \item Performance evaluation across multiple SDGFTS models and datasets
  \end{itemize}

  By analyzing the results from our MLflow experiments and applying our unified benchmark, 
  we aim to provide clear guidelines for evaluating SDGFTS models and establish a standard 
  for future research in this domain.

  \subsection{Limitations in the SDGFTS Literature}
    Among other things, we observed that there is currently no universal, generally accepted approach 
    to evaluating synthetic time series\cite{stenger24survey}; this issue extends beyond FTS to generative 
    frameworks like GANs as a whole \cite{wang20gan}. 
    Many evaluation measures are insufficiently defined and lack public implementations, making reuse 
    and reproduction troublesome and error-prone \cite{stenger24survey}. This presents a challenge 
    unique to the generation task compared to areas like time series forecasting or classification 
    \cite{stenger24survey}.
    Hence, future research would immensely benefit from a widely accepted, reasonably sized set of 
    qualified measures for the central evaluation criteria.
  \subsection{Our Contributions}
    To put it in simple terms, we are adopting the time series evaluation taxonomy outlined from 
    Stenger et al. \cite{stenger24survey}, develop a comprehensive and unified SDG benchmark expanded from 
    the works of Ang et al. \cite{Ang23} in specifics to FTS by incorporating a utility evaluation 
    framework inspired by Boursin et al. \cite{Boursin22} through portfolio evaluations generated by 
    a deep hedger.
    Our key contributions include:
    \begin{enumerate}[label=\textbf{[C\arabic*]}]
      \item \label{contribute:1} \textbf{Consolidated Evaluation Framework:}
        We systematically review and consolidate evaluation methods from leading papers (models and surveys)
        in SDG and financial time series \cite{stenger24survey, Ang23, Boursin22}, creating a 
        comprehensive assessment toolkit based on established practices.

      \item \label{contribute:2} \textbf{Unified Statistical and Utility Measures:}
        For the first time, we integrate both statistical fidelity metrics and practical 
        utility measures in a single SDGFTS benchmark, providing a more complete 
        evaluation of synthetic data quality.

      \item \label{contribute:3} \textbf{Open Benchmark Platform:}
        We deliver an open-source benchmark framework for the research community, 
        aiming to establish a gold standard for SDGFTS model evaluation and 
        comparison.
    \end{enumerate}


\section{Preliminaries}
  \subsection{Problem Definition}
  Let us formally define the Synthetic Data Generation problem for Financial Time Series. Given a financial time series $\textbf{X} = (\textbf{x}_1,...,\textbf{x}_R)^T$ with $R$ individual series represented as an $L$-dimensional vector $\textbf{x}_i=(x_{i1},...,x_{iL})$, where $x_{ij}$ denotes the $j$-th point of time series $\textbf{x}_i$. Let $p(\textbf{x}_1,...,\textbf{x}_R)$ denote the real distribution of the given time series $\textbf{X}$. The objective of SDG for FTS 
  is to generate a synthetic time series 
  $\mathbf{\hat{X}} = (\mathbf{\hat{x}}_1, \ldots, \mathbf{\hat{x}}_n)^{T}$ such that its distribution $p(\mathbf{\hat{x}}_1, \ldots, \mathbf{\hat{x}}_n)$ approximates $p(\textbf{x}_1,...,\textbf{x}_R)$, while preserving the key properties exhibited in the real financial time series, such as autocorrelation structure, volatility clustering, and distributional moments (mean, variance, skewness, kurtosis).


  \subsection{Scope of Project}
    \subsubsection{Scope of Methods}
      Our benchmark encompasses a diverse range of SDGFTS methods, from traditional parametric approaches 
      to modern deep learning architectures. We selected models based on three main criteria:
      (1) proven industry adoption, (2) research community acceptance, and (3) recent methodological innovations.

      Our evaluation includes classical parametric models, which rely on predefined statistical distributions 
      and assumptions about the underlying data generation process. These models have been extensively used 
      in financial institutions for their interpretability and theoretical foundations. 

      We also evaluate modern non-parametric approaches, particularly deep learning-based models that learn 
      the data distribution directly from observations without assuming a specific form. These include 
      generative adversarial networks, variational autoencoders, and diffusion models, representing the 
      cutting edge in synthetic data generation.

      \subsubsection{Scope of Datasets}
      We evaluate models on diverse financial datasets spanning:
      \begin{itemize}
        \item Stock market indices (S\&P 500, NASDAQ, TSX)
        \item Individual stock prices from major exchanges
        \item Forex trading pairs
      \end{itemize}

      \subsubsection{Scope of Evaluation Taxonomical Criteria}
      Our evaluation framework follows the taxonomy structure proposed by Stenger et al. \cite{Stenger24}, 
      incorporating various evaluation measures from different papers in the field.

      We systematically integrate evaluation metrics from multiple sources while maintaining this 
      structured taxonomical approach, ensuring comprehensive coverage of all critical aspects of 
      SDGFTS evaluation and setting a standard for future research in time series SDG as a whole.

\section{Overview of Synthetic FTS Methods}
    We categorize synthetic financial time series generation methods into two main families: classical parametric (stochastic) models, which rely on explicit assumptions about the data-generating process, and modern non-parametric (deep learning) models, which learn complex dependencies directly from data without predefined functional forms. The following subsections provide a detailed overview of the methods included in our benchmark.


  \subsection{Classical Stochastic (Parametric) Methods}
    Parametric models assume a specific functional form for the data-generating process, characterised by 
    a finite set of parameters. These models are interpretable and analytically tractable, allowing for 
    closed-form solutions in many cases. However, they may struggle to capture complex stylised facts 
    or high-dimensional dependencies beyond their structural assumptions. Below we outline several 
    widely-used parametric models in FTS.\\

      \begin{table}[h]
        \centering
        \caption{Common notation and variables.}
        \label{tab:notation}
        \begin{tabular}{ll}
        \toprule
        \textbf{Symbol} & \textbf{Description} \\
        \midrule
        \( S_t \) & Asset price \\
        \( X_t = \ln S_t \) & Log-price \\
        \( r_t = \ln(S_t / S_{t-1}) \) & Log-return \\
        \( \mu \) & Drift \\
        \( \sigma \) & Volatility \\
        \( W_t \) & Standard Brownian motion \\
        \( \Delta t \) & Discrete time step \\
        \bottomrule
        \end{tabular}
      \end{table}
        
        

      \begin{enumerate}[label=\textbf{[A\arabic*]}]
        \item \label{alg:gbm} \textbf{Geometric Brownian Motion (GBM).} 
        The price process follows a continuous-time stochastic process with constant drift and volatility. 
        Model-specific parameters are \(\mu\) and \(\sigma\).
        \begin{equation}
        dS_t = \mu S_t\,dt + \sigma S_t\,dW_t.
        \end{equation}

        \item \label{alg:ou} \textbf{Ornstein–Uhlenbeck (OU).} 
        A mean-reverting Gaussian process for log-prices or spreads. Model-specific parameters are 
        \(\theta>0\) (reversion rate), \(\mu\) (long-term mean), and \(\sigma\) (volatility).
        \begin{equation}
        dX_t = \theta(\mu - X_t)\,dt + \sigma\,dW_t.
        \end{equation}

        \item \label{alg:mjd} \textbf{Merton Jump Diffusion (MJD).} Extends GBM by incorporating 
        normally-distributed jumps. Model-specific parameters are \(\lambda\) (jump intensity), 
        \(\mu_j\) and \(\sigma_j\) (mean and standard deviation of jump sizes).
        \begin{equation}
        \frac{dS_t}{S_t} = \mu\,dt + \sigma\,dW_t + J\,dN_t,\\
        \end{equation}
        where $J \sim \mathcal N(\mu_j, \sigma_j^2),\quad N_t\sim\mathrm{Poisson}(\lambda t)$.


        \item \label{alg:dejd} \textbf{Double-Exponential Jump Diffusion (DEJD).} 
        A jump-diffusion model with asymmetric double-exponential jump sizes. Model-specific 
        parameters are \(p\) (probability of upward jump) and \(\eta_1, \eta_2\) 
        (exponential parameters for upward/downward jumps).
        \begin{align}
          \frac{dS_t}{S_t} &= \mu\,dt + \sigma\,dW_t + Y\,dN_t,\\
          Y &=
          \begin{cases}
          \mathrm{Exp}(\eta_1), &\text{with probability } p,\\[4pt]
          -\mathrm{Exp}(\eta_2), &\text{with probability } 1-p.
          \end{cases} \nonumber
        \end{align}

        \item \label{alg:garch} \textbf{GARCH(1,1).} 
        A discrete-time conditional heteroskedastic model for returns. Model-specific parameters 
        are \(\omega>0,\ \alpha\ge0,\ \beta\ge0\) (GARCH coefficients) and \(\mathcal D\) 
        (innovation distribution, usually standard Normal or Student-t). Captures volatility clustering 
        via time-varying conditional variance.
        \begin{align}
          r_t &= \sigma_t z_t,\quad z_t\sim\mathcal D,\\
          \sigma_t^2 &= \omega + \alpha r_{t-1}^2 + \beta \sigma_{t-1}^2. \nonumber
        \end{align}
        \item \label{alg:blockbootstrap} \textbf{Block Bootstrap.} 
        A resampling technique that preserves short-range dependence by sampling contiguous blocks of returns. 
        Although not parametric, it is included here as it is a statistical model.
      \end{enumerate}


  \subsection{Deep Learning (Non-parametric) Methods}
    Non-parametric models, in this context, refer to generative (often implicit) models that do not assume a 
    fixed parametric form for the underlying data-generating process. Instead, they learn latent structures 
    directly from data through flexible architectures such as neural networks, enabling the modeling of highly 
    nonlinear, high-dimensional, and multi-modal dependencies (e.g., multivariate financial time series with 
    interactions across assets, time horizons, and features). While these models trade off interpretability and 
    analytical tractability, they offer substantially greater expressive power and flexibility. 
    This expressiveness, however, comes at the cost of increased data requirements, more intensive 
    hyperparameter tuning, and higher computational demands. 

    We categorize these approaches into three major families of deep generative models: 
    Generative Adversarial Networks (GANs), Variational Autoencoders (VAEs), and Diffusion Models.
    

  \subsubsection{Diffusion Models}
    Diffusion models (or score-based generative models) have emerged as powerful methods for modeling complex 
    data distributions via a forward (noise-adding) process and a learned reverse (denoising) process 
    \cite{sohl2015deep, ho2020denoising}. In the forward direction, one gradually corrupts a data sample 
    \(x_0\) into noise via a Markov chain:
   
    \begin{align}
      q(x_{1:T}\mid x_0) &= \prod_{t=1}^T q(x_t \mid x_{t-1}), \\
      q(x_t \mid x_{t-1}) &= \mathcal{N}\left(x_t; \sqrt{1-\beta_t}\,x_{t-1}, \beta_t I\right). \nonumber
    \end{align}
   
    
    The reverse (generative) model is parameterized as
    \[
      p_\theta(x_{t-1} \mid x_t) = \mathcal{N}\bigl(x_{t-1}; \mu_\theta(x_t, t), \Sigma_\theta(x_t, t)\bigr),
    \]
    where \(\mu_\theta\) and \(\Sigma_\theta\) are often expressed via a neural network that estimates the 
    score \(\nabla_{x_t} \log q(x_t)\). A common training objective is the simplified denoising score-matching 
    loss:
    \[
      \mathbb{E}_{t, x_0, \epsilon}\Bigl[ \bigl\| \epsilon - \epsilon_\theta(x_t, t)\bigr\|^2 \Bigr],
    \]
    where \(x_t = \sqrt{\bar\alpha_t}\,x_0 + \sqrt{1-\bar\alpha_t}\,\epsilon\).
    
    In finance/time-series, diffusion models have recently been adapted to generate synthetic asset paths by 
    converting multivariate time series into suitable representations (e.g. wavelet-transformed images) and 
    then sampling via reverse diffusion \cite{Tanaka25}. For instance, 
    Takahashi and Mizuno shows that such an approach can 
    reproduce key statistical properties of financial data (e.g.\ volatility clustering, tail behavior) 
    \cite{Takahashi24}.  
    Another relevant work is “A Financial Time Series Denoiser Based on Diffusion Model”, which shows 
    how diffusion can help improve downstream predictability and reduce noise in financial signals 
    \cite{wang24denoiser}.  
    
    \subsubsection{Variational Autoencoders (VAEs)}
    Variational Autoencoders \cite{kingma2013auto, rezende2014stochastic} are latent-variable generative 
    models that posit a probabilistic encoder \(q_\phi(z \mid x)\) and decoder \(p_\theta(x \mid z)\). 
    One maximizes the evidence lower bound (ELBO):
    \[
    \mathcal{L}_{\text{VAE}} = \mathbb{E}_{q_\phi(z \mid x)}\bigl[\log p_\theta(x \mid z)\bigr] - \mathrm{KL}\bigl(q_\phi(z \mid x)\,\Vert\,p(z)\bigr).
    \]
    Often one uses a standard Gaussian prior \(p(z) = \mathcal{N}(0, I)\). The encoder–decoder structure allows sampling by first drawing \(z \sim p(z)\), then generating \(x \sim p_\theta(x \mid z)\).
    
    In financial time-series generation, specialized versions such as the Time-Causal VAE (TC-VAE) have been proposed to enforce causality in the encoding/decoding of temporal data. For instance, Acciaio et al.\ (2024) propose TC-VAE, which ensures that generated paths respect temporal causality and show that the model reproduces stylized facts (e.g., heavy tails, volatility clustering) on real markets \cite{acciaio2024tcvae}. Because VAEs provide a principled latent representation, they are useful in finance to model underlying latent drivers of asset dynamics and to sample coherent trajectories.
    
    \subsubsection{Generative Adversarial Networks (GANs)}
    Generative Adversarial Networks \cite{goodfellow2014generative} frame generation as a two-player game between a generator \(G(z)\) and a discriminator \(D(x)\). The classical minimax objective is
    \[
    \min_G \max_D \; \mathbb{E}_{x \sim p_{\text{data}}} \bigl[\log D(x)\bigr] + \mathbb{E}_{z \sim p(z)}\bigl[\log (1 - D(G(z)))\bigr].
    \]
    In practice, many variants (e.g.\ Wasserstein GAN, Least Squares GAN) are used to improve stability.
    
    In financial data synthesis, GANs have been widely used to generate realistic synthetic time series. For example, GAN-based financial data generation models (often using WGAN or improvements) show that one can improve the authenticity and predictive ability of generated financial statements or time series \cite{qi2025gan_financial}. Another example is VRNNGAN, which uses a recurrent VAE as the generator and a recurrent discriminator to capture temporal dependencies in synthetic sequence generation \cite{lee2022vrnngan}. GANs are relevant in finance because they can capture complex joint distributions and dependencies in multivariate time-series without requiring explicit likelihood models.
    
  \subsection{Miscellaneous}
  This section consists of models that are classified as neither parametric nor non-parametric.
  \begin{enumerate}[label=\textbf{[A\arabic*]}]
        \item \label{alg:blockbootstrap} \textbf{Block Bootstrap (non-parametric).} 
        A resampling technique that preserves short-range dependence by sampling contiguous blocks of returns. 
        Although not parametric, it is included here for completeness and baseline comparison.
      \end{enumerate}


\section{Stonk Bench Architecture}
  \subsection{Datasets and Preprocessing}
    The preprocessing procedure builds upon the TSGBench pipeline, which standardizes segmentation, 
    normalization, and train/test splitting of multivariate time series. Several modifications were 
    introduced to better accommodate financial time series and stochastic models.

    \subsubsection{Data type.} Stock price data (e.g., AAPL) at daily frequency with OHLC columns
    (Open, High, Low, Close) are used. Let \(X \in \mathbb{R}^{L \times N}\) denote a multivariate time series 
    with \(N=4\) channels and length \(L\).

    \subsubsection{Transformations.} Raw prices are converted to log-returns per channel via

    \[ r_t = \log P_t - \log P_{t-1} = \log(P_t / P_{t-1}) \,, \quad r_t \in \mathbb{R}^N, \]
    
    yielding a transformed series of length \(L-1\).

    \subsubsection{Model-specific preprocessing.}  
    The preprocessing follows the general TSGBench framework, with the following adaptations:

    \begin{itemize}
        \item Log returns are used directly instead of TSGBench's min-max normalization, preserving 
        financial properties such as heavy tails and volatility clustering.
        \item For parametric models, the transformed series is divided into contiguous training, 
        validation, and test segments with ratios \((1-\alpha-\beta):\alpha:\beta\), where 
        \(\alpha=0.1\) and \(\beta=0.1\). This ensures that parametric models fit parameters directly 
        to contiguous historical data.
        \item For non-parametric models, overlapping sliding windows \(\mathcal{W} \in \mathbb{R}^{R \times L \times N}\) 
        with stride 1 are extracted. Unlike TSGBench, which employs an autocorrelation-based hard-coded 
        window of 125 (often yielding a size of 1 for stock data due to fast autocorrelation decay), 
        the window length \(L\) is determined via the partial autocorrelation function (PACF). Specifically, 
        PACF is computed for each channel up to \(\lfloor 10\log_{10} n \rfloor\) lags, and the lag corresponding 
        to the maximum significant PACF peak across channels is selected, resulting in \(L=13\). This approach 
        captures relevant temporal dependencies within the data.
        \item The dataset is split into train, validation, and test sets while preserving temporal order to prevent future information from leaking into the past. Only the training set is shuffled during model training to improve convergence. In contrast, TSG-Bench shuffles time series samples before splitting, which introduces data leakage. Our approach ensures strictly forward-looking evaluation for realistic forecasting.

        proper evaluation, consistent with TSGBench.
    \end{itemize}

    \subsubsection{Data loaders.} For non-parametric models, mini-batches are generated from the windowed 
    data \(\mathcal{W}\) using independent fixed seeds for training, validation, and test sets, 
    enabling reproducible epoch-wise evaluation. Training batches are shuffled, while validation and test 
    sets maintain sequential ordering. To ensure fair comparison, parametric models generate sequences 
    of identical length \(L\) and matching sample count to the windowed data used by non-parametric models.


  \subsection{Statistical Evaluation Measures}
    All metrics below are applied \textbf{per channel}, i.e., each univariate time series (OHLC column) is evaluated independently. For multivariate data, results are reported as channel-wise statistics (e.g., mean or distribution across channels).
    
    \subsubsection{Feature-based Distance Measures}
    These metrics quantify how well synthetic data captures key marginal and second-order properties of real data.
    
    \begin{enumerate}[label=\textbf{[M\arabic*]}]
      \item \label{metric:mdd} \textbf{Marginal Distribution Difference (MDD):}  
      Distance between real and synthetic marginals (e.g., 1-Wasserstein distance) per channel:
      \[
      \begin{aligned}
        \mathrm{MDD}_j &= W_1\big(\widehat{F}^j_{real}, \widehat{F}^j_{synth}\big), \\
        W_1(F, G) &= \int_0^1 \big|F^{-1}(u) - G^{-1}(u)\big|\,du,
      \end{aligned}
      \]
      where \(\widehat{F}^j_{real}\) and \(\widehat{F}^j_{synth}\) are empirical CDFs of channel \(j\).
    
      \item \label{metric:md} \textbf{Mean Difference (MD):}  
      Absolute difference of per-channel means:
      \[
        \mathrm{MD}_j = \big|\mu^j_{real} - \mu^j_{synth}\big|.
      \]
    
      \item \label{metric:sdd} \textbf{Standard Deviation Difference (SDD):}  
      Absolute difference of per-channel standard deviations:
      \[
        \mathrm{SDD}_j = \big|\sigma^j_{real} - \sigma^j_{synth}\big|.
      \]
    
      \item \label{metric:kd} \textbf{Kurtosis Difference (KD):}  
      Absolute difference of per-channel excess kurtosis:
      \[
      \begin{aligned}
      \mathrm{KD}_j &= \big|\kappa^{\mathrm{ex}}(X^j_{\text{real}}) - \kappa^{\mathrm{ex}}(X^j_{\text{synth}})\big|, \\
      \kappa^{\mathrm{ex}}(X^j) &= \frac{\mathbb{E}[(X^j - \mu^j)^4]}{(\sigma^j)^4} - 3.
      \end{aligned}
      \]
    
    
      \item \label{metric:acd} \textbf{AutoCorrelation Difference (ACD):}  
      Absolute difference in lag-1 autocorrelation per channel:
      \[
      \begin{aligned}
        \rho^j(1) &= \frac{\sum_{t=2}^{L} (x^j_t - \bar{x}^j)(x^j_{t-1} - \bar{x}^j)}
                       {\sum_{t=1}^{L} (x^j_t - \bar{x}^j)^2}, \\
        \mathrm{ACD}_j &= \big|\rho^j_{real}(1) - \rho^j_{synth}(1)\big|.
      \end{aligned}
      \]
    \end{enumerate}
    
    \subsubsection{Visualization Methods}
    Visual tools complement numeric metrics for face-validity and communication \cite{Ang23}.
    \begin{enumerate}[label=\textbf{[M\arabic*]}, resume]
      \item \label{metric:tsne} \textbf{t-SNE:}  
      2D embeddings of window-level features per channel for real vs. synthetic overlap and cluster structure.
    
      \item \label{metric:dist} \textbf{Distribution:}  
      Channel-wise histograms of returns to visualize empirical distribution.
    \end{enumerate}
    
    \subsubsection{Diversity Metrics}
    Measures that prevent mode collapse, computed per channel. Pairwise distances only consider the same channel across samples.
    
    \begin{enumerate}[label=\textbf{[M\arabic*]}, resume]
      \item \label{metric:ed} \textbf{Intra-Class Euclidean Distance (ICD-ED):}  
      \[
        \mathrm{ICD\text{-}ED}^j = \frac{2}{R(R-1)} \sum_{1 \le a < b \le R} \left\| \textbf{x}_a^j - \textbf{x}_b^j \right\|_2,
      \]
      where \(\textbf{x}_a^j \in \mathbb{R}^{L}\) is the \(j\)-th channel of the \(a\)-th sample.
    
      \item \label{metric:dtw} \textbf{Intra-Class Dynamic Time Warping (ICD-DTW):}  
      \[
        \mathrm{ICD\text{-}DTW}^j = \frac{2}{R(R-1)} \sum_{1 \le a < b \le R} d_{\mathrm{DTW}}\bigl(\textbf{x}_a^j, \textbf{x}_b^j\bigr).
      \]
    \end{enumerate}
    
    \subsubsection{Efficiency Assessment}
    \begin{enumerate}[label=\textbf{[M\arabic*]}, resume]
      \item \label{metric:gentime} \textbf{Generation Time:}  
      Wall-clock time to generate \(S\) samples:
      \[
        T_{\mathrm{gen}} = t_1 - t_0 \quad (\text{seconds for } S \text{ samples}).
      \]
    \end{enumerate}
    
    \subsubsection{Stylized Facts Verification}
    Canonical stylized facts, measured per channel:
    
    \begin{enumerate}[label=\textbf{[M\arabic*]}, resume]
      \item \label{metric:heavytails} \textbf{Heavy Tails:}  
      Per-channel excess kurtosis:
      \[
        \kappa^{\mathrm{ex}}(X^j) = \frac{\mathbb{E}[(X^j - \mathbb{E}[X^j])^4]}{(\mathrm{Var}[X^j])^2} - 3.
      \]
    
      \item \label{metric:autocorr} \textbf{Return Autocorrelation:}  
      Lag-1 autocorrelation per channel, should be close to zero for liquid assets.
    
      \item \label{metric:volclustering} \textbf{Volatility Clustering:}  
      Lag-1 autocorrelation of squared or absolute returns per channel:
      \[
      \begin{aligned}
      \rho^j_{x^2}(1) &= \mathrm{Corr}\big((x^j_t)^2, (x^j_{t-1})^2\big), \\
      \rho^j_{|x|}(1) &= \mathrm{Corr}\big(|x^j_t|, |x^j_{t-1}|\big).
      \end{aligned}
      \]
        
    \end{enumerate}
    
  
  \subsection{Utility Evaluation Measures: Deep Hedging}
    Utility measures are critical for evaluating synthetic data beyond statistical metrics, 
    as they assess the practical value of generated data in real-world financial applications \cite{Boursin22}. 
    Our benchmark incorporates deep hedging as a utility measure for several key reasons:

    \begin{enumerate}[label=\textbf{[U\arabic*]}]
      \item \label{utility:application}\textbf{Real-world Application Testing:}
      Deep hedging provides a concrete way to evaluate how synthetic data performs in 
      actual financial tasks, particularly in derivatives pricing and risk management.

      \item \label{utility:industry}\textbf{Industry-relevant Metrics:}
      By comparing hedging strategies trained on synthetic versus real data, we can 
      assess the practical utility of synthetic data through metrics that matter to 
      financial practitioners, such as replication errors and hedging performance.

      \item \label{utility:validation}\textbf{Model Robustness Validation:}
      Deep hedging helps verify if synthetic data maintains the complex relationships 
      and market dynamics necessary for developing reliable trading strategies.
    \end{enumerate}

    \subsubsection{Deep Hedger Problem Defintion}
      We consider a continuous-time financial market defined on a probability space 
      $(\Omega, \mathcal{F}, \mathbb{P})$, over a finite time horizon $0 < T < \infty$, 
      equipped with a filtration $\mathcal{F} = (\mathcal{F}_t)_{0 \leq t \leq T}$ 
      that represents the evolution of information through time.  
      The market consists of $d + 1$ tradable assets $S = (S^0, \dots, S^d)$, where 
      $S_t^j$ denotes the price of asset $j$ at time $t$. For simplicity, we assume a 
      zero interest rate environment.

      We study the hedging problem associated with a contingent claim delivering a payoff 
      $g(S_T)$ at maturity $T$, where $S_T$ represents the terminal value of the underlying 
      asset vector.  
      The hedging strategy is implemented at a discrete set of times 
      $0 = t_0 < t_1 < \dots < t_{N-1} < t_N = T$.  
      A \textit{self-financing portfolio} is described by a $d$-dimensional $\mathcal{F}_t$
      -adapted process $\Delta_t$, and its terminal wealth, denoted $X_T^{\Delta, p}$, 
      is given by
      \begin{equation}
          X_T^{\Delta, p} = p + \sum_{i=1}^{d} \sum_{j=0}^{N-1} \Delta_{t_j}^i 
          \left( F_{t_{j+1}}^i - F_{t_j}^i \right),
      \end{equation}
      where $p \in \mathbb{R}$ represents the initial premium.

      The objective is to determine the optimal premium and trading strategy 
      $(p^{\text{opt}}, \Delta^{\text{opt}})$ that minimize the expected squared hedging error:
      \begin{equation}
          (p^{\text{opt}}, \Delta^{\text{opt}}) 
          = \operatorname*{Argmin}_{p, \Delta} 
          \mathbb{E} \left[ \big( X_T^{\Delta} - g(S_T) \big)^2 \right].
      \end{equation}

      To address this optimization problem, we employ the global approach proposed by 
      Fécamp \textit{et al.} (2020), chosen for its computational efficiency and scalability.  
      In this framework, the control policy $\Delta$ is approximated by a feed-forward neural 
      network—referred to as a \textit{deep hedger}—which jointly learns the optimal trading 
      strategy and corresponding premium through end-to-end training.
    \subsubsection{Deep Hedger Architecture}
      \begin{table}[h]
        \centering
        \caption{Notations for Deep Hedger Architecture.}
        \label{tab:deephedger-notation}
        \begin{tabular}{ll}
        \toprule
        \textbf{Symbol}     & \textbf{Description} \\
        \midrule
        \( D \)             & Time-series dataset             \\
        \( T \)             & Task                            \\
        \( S \)             & Score                           \\
        \( A \)             & Deep hedger algorithm           \\
        \( \mathcal{A}_n \) & Set of n deep hedger algorithms \\
        \( M \)             & Deep hedger model               \\
        \( \mathcal{M}_{TS} \)   & (SDGFTS) model                  \\
        \bottomrule
        \end{tabular}
      \end{table}

      Our deep hedger architecture centeres around the Train-Synthetic-Test-Real (TSTR)
      evaluation framework \cite{Leznik21}, adapted for financial time series generation.
      We begin with our real FTS dataset \( D\)\textemdash partitioned to a training, validation, 
      and test set respectively as
      \( D_r = \{D_r^{train}, D_r^{validate}, D_r^{test}\} \)\textemdash along with a specific 
      hedging task \(T\) and a performance score \(S\) to evaluate hedging effectiveness, 
      to which we will go indepth in the next subsubsection.
      We consider a deep hedger algorithm \(A\) (a 5-layer perceptron) that takes as 
      input the dataset \(D_r\) and task \(T\),

      In the context of SDGFTS evaluation, we consider a SDGFTS model \( \mathcal{M}_{TS} \) 
      trained on \( D_r^{train} \) and validated with \( D_r^{validate} \) to generate 
      synthetic FTS data \( D_g \).

      Now we can train a deep hedger algorithm \( A \) on the datasets \( D_r^{train} \) and 
      \( D_g^{train} \) respectively, resulting in two deep hedger models
      \( M_r \) and \( M_g \).

      Finally, we evaluate both models on the real test set \( D_r^{test} \) to obtain the
      corresponding performance scores \( s_r \) and \( s_g \).

      The goal of the deep hedger portfolio evaluation is to evaluate the effectiveness of the
      model \( \mathcal{M}_{TS} \) with downstream tasks (deep hedging) by comparing the scores 
      \( s_r \) and \( s_g \). The model \( \mathcal{M}_{TS} \) is considered effective if the 
      the results yielded similar scores, i.e. \( s_r \approx s_g \) \textemdash preferably having 
      higher performance with results of \( s_g > s_r \) \textemdash indicating that the synthetic 
      data generated by \( \mathcal{M}_{TS} \) is useful for training deep hedger models 
      \cite{Stenger24}.

    \subsubsection{Deep Hedger Extentions}
    We proposed two extensions to the standard deep hedger architecture to better suit our 
    benchmark need that drew inspiration from recent literature \cite{Stenger24}.
      \begin{enumerate}[label=\textbf{[E\arabic*]}]
        \item \label{ext:augment} \textbf{Augmented Testing \cite{Stenger24}}  
        To better evaluate the performance of deep hedgers trained on synthetic data, 
        we train our deep hedger \( A \) on a new dataset \( D_{aug} \) that combines
        both real training data \( D_r^{train} \) and synthetic data \( D_g\). In other words,
        we traid model \( M_g \) on \( D_{aug} \) where \( D_{aug} := D_r^{train} \cup D_g\).
        Then we proceed to evaluate scores between models\( M_g \) and \( M_r \) 
        on the real test set, where \( M_r \) i s still trained on only real data \( D_r^{train} \).


        \item \label{ext:algcomparsion} \textbf{Algorithm Comparison \cite{Lin21}}  
        Another extension is to compare multiple deep hedger algorithms indicating to what 
        degree each algorithm performs equally on the generated data relative to the other
        algorithms, compared to their performance on real data.

        Given a set of \( n \) deep hedger algorithms 
        \( \mathcal{A}_n = \{ A_1, \ldots, A_n \} \), we train each algorithm 
        \( A_i \in \ mathcal{A}_n \) on both real training data \( D_r^{train} \) and
        synthetic data \( D_g^{train} \) respectively, resulting in two sets of deep 
        hedger models \(\{M_r^1, M_r^2, \ldots, M_r^n\} \) and \( \{M_g^1, M_g^2, \ldots, M_g^n\} \)
        and two ordered setes of performance scores \( s_r := \{s_r^1, s_r^2, \ldots, s_r^n\} \) and 
        \( s_g := \{s_g^1, s_g^2, \ldots, s_g^n\} \) respectively.

        Finally we compare the relative performance of each algorithm on real data versus synthetic data
        by computing the Spearman's rank correlation coefficient \( r_S \) \cite{Lin21} between 
        the two score sets \( s_r \) and \( s_g \):
        \begin{equation}
          r_S = 1 - \frac{6 \sum_{i=1}^{n} ( \mathrm{rank}(s_r^i) - \mathrm{rank}(s_g^i) )^2}{n(n^2 - 1)},
        \end{equation}
        where \( \mathrm{rank}(s_r^i) \) and \( \mathrm{rank}(s_g^i) \) denote the ranks of
        scores \( s_r^i \) and \( s_g^i \) within their respective sets.

        We interpret a high positive correlation (i.e., \( r_S \) close to 1) as an indication that
        the synthetic data  generated by \( mathcal{M}_{TS} \) preserves the relative performance 
        of different deep hedger algorithms.
      \end{enumerate}
    % \subsubsection{Deep Hedger Portfolio Evaluation}
    %   In the question of how to effectively evaluate the deep hedging utility of synthetic 
    %   financial time series data, we wish to choose scores \( S \) that reflect the practical
    %   utility of the data in real-world hedging tasks.
    %   We propose the following evaluation scores:
    %   \begin{enumerate}[label=\textbf{[M\arabic*]}, resume]
    %     \item \label{metric:replicationerror} \textbf{Replication Error}
    %     \item \label{metric:erm} \textbf{ERM}
    %     \item \label{metric:cvar} \textbf{CVaR}
    %   \end{enumerate}

\section{Results and Analysis}
  \subsection{Statistical Evaluation Results}
  We report per-model, per-channel metrics. Copy values from the generated JSON (e.g., \texttt{complete\_evaluation.json}) into the following templates.

  \begin{table*}[h]
    \centering
    \caption{Fidelity metrics by model and channel (lower is better for differences). Channels C1--C4 correspond to dataset channels (e.g., Open, High, Low, Close).}
    \begin{tabular}{l|cccc|cccc|cccc|cccc}
      \toprule
      & \multicolumn{4}{c|}{MDD} & \multicolumn{4}{c|}{MD} & \multicolumn{4}{c|}{SDD} & \multicolumn{4}{c}{KD} \\
      Model & C1 & C2 & C3 & C4 & C1 & C2 & C3 & C4 & C1 & C2 & C3 & C4 & C1 & C2 & C3 & C4 \\
      \midrule
      GBM &  &  &  &  &  &  &  &  &  &  &  &  &  &  &  &  \\
      OU\_Process &  &  &  &  &  &  &  &  &  &  &  &  &  &  &  &  \\
      MJD &  &  &  &  &  &  &  &  &  &  &  &  &  &  &  &  \\
      GARCH11 &  &  &  &  &  &  &  &  &  &  &  &  &  &  &  &  \\
      DEJD &  &  &  &  &  &  &  &  &  &  &  &  &  &  &  &  \\
      BlockBootstrap &  &  &  &  &  &  &  &  &  &  &  &  &  &  &  &  \\
      TimeGAN &  &  &  &  &  &  &  &  &  &  &  &  &  &  &  &  \\
      QuantGAN &  &  &  &  &  &  &  &  &  &  &  &  &  &  &  &  \\
      TimeVAE &  &  &  &  &  &  &  &  &  &  &  &  &  &  &  &  \\
      \bottomrule
    \end{tabular}
  \end{table*}

  \begin{table*}[h]
    \centering
    \caption{Temporal and diversity metrics by model and channel (lower is better for differences; higher is better for diversity distances).}
    \begin{tabular}{l|cccc|cccc|cccc}
      \toprule
      & \multicolumn{4}{c|}{ACD} & \multicolumn{4}{c|}{ICD-ED} & \multicolumn{4}{c}{ICD-DTW} \\
      Model & C1 & C2 & C3 & C4 & C1 & C2 & C3 & C4 & C1 & C2 & C3 & C4 \\
      \midrule
      GBM &  &  &  &  &  &  &  &  &  &  &  &  \\
      OU\_Process &  &  &  &  &  &  &  &  &  &  &  &  \\
      MJD &  &  &  &  &  &  &  &  &  &  &  &  \\
      GARCH11 &  &  &  &  &  &  &  &  &  &  &  &  \\
      DEJD &  &  &  &  &  &  &  &  &  &  &  &  \\
      BlockBootstrap &  &  &  &  &  &  &  &  &  &  &  &  \\
      TimeGAN &  &  &  &  &  &  &  &  &  &  &  &  \\
      QuantGAN &  &  &  &  &  &  &  &  &  &  &  &  \\
      TimeVAE &  &  &  &  &  &  &  &  &  &  &  &  \\
      \bottomrule
    \end{tabular}
  \end{table*}

  \begin{table*}[h]
    \centering
    \caption{Stylized facts by model and channel (differences reported real minus synthetic unless noted).}
    \begin{tabular}{l|cccc|cccc|cccc|cccc}
      \toprule
      & \multicolumn{4}{c|}{Heavy Tails (diff)} & \multicolumn{4}{c|}{Autocorr Raw (diff)} & \multicolumn{4}{c|}{Volatility Clustering (diff)} & \multicolumn{4}{c}{Long Memory (diff)} \\
      Model & C1 & C2 & C3 & C4 & C1 & C2 & C3 & C4 & C1 & C2 & C3 & C4 & C1 & C2 & C3 & C4 \\
      \midrule
      GBM &  &  &  &  &  &  &  &  &  &  &  &  &  &  &  &  \\
      OU\_Process &  &  &  &  &  &  &  &  &  &  &  &  &  &  &  &  \\
      MJD &  &  &  &  &  &  &  &  &  &  &  &  &  &  &  &  \\
      GARCH11 &  &  &  &  &  &  &  &  &  &  &  &  &  &  &  &  \\
      DEJD &  &  &  &  &  &  &  &  &  &  &  &  &  &  &  &  \\
      BlockBootstrap &  &  &  &  &  &  &  &  &  &  &  &  &  &  &  &  \\
      TimeGAN &  &  &  &  &  &  &  &  &  &  &  &  &  &  &  &  \\
      QuantGAN &  &  &  &  &  &  &  &  &  &  &  &  &  &  &  &  \\
      TimeVAE &  &  &  &  &  &  &  &  &  &  &  &  &  &  &  &  \\
      \bottomrule
    \end{tabular}
  \end{table*}

  \begin{table}[h]
    \centering
    \caption{Non-stationarity (diff) and generation time per model.}
    \begin{tabular}{l|cccc|c}
      \toprule
      & \multicolumn{4}{c|}{Non-stationarity (diff)} & Gen. Time (500) \\
      Model & C1 & C2 & C3 & C4 & seconds \\
      \midrule
      GBM &  &  &  &  &  \\
      OU\_Process &  &  &  &  &  \\
      MJD &  &  &  &  &  \\
      GARCH11 &  &  &  &  &  \\
      DEJD &  &  &  &  &  \\
      BlockBootstrap &  &  &  &  &  \\
      TimeGAN &  &  &  &  &  \\
      QuantGAN &  &  &  &  &  \\
      TimeVAE &  &  &  &  &  \\
      \bottomrule
    \end{tabular}
  \end{table}

  \subsection{Utility Evaluation Results}
  Results of the deep hedging evaluation (replication error, P\&L distribution, risk-adjusted metrics) can be summarized in tables analogous to the 
  above once computed.

  \subsection{Comprehensive Model Comparison}
  Provide radar plots or scorecards that combine normalized metrics (fidelity, diversity, stylized facts, efficiency) into composite ranks per task.

  \subsection{Ranking Analysis}
  Discuss trade-offs: fidelity vs. diversity, training time vs. quality, and sensitivity to window length and sample count.

\section{Conclusion and Future Work}
  In our research, we recognize the existing limitations in the evaluation of synthetic time series, 
  particularly the absence of a universally accepted framework. To address this, we adopt the 
  evaluation taxonomy proposed by Stenger et al. and expand upon it to create a comprehensive benchmark 
  specifically tailored for Synthetic Data Generation for Financial Time Series (SDGFTS). 
  Our contributions include the development of a consolidated evaluation framework that systematically 
  reviews and integrates various assessment methods from leading studies in the field. 
  This approach not only enhances the rigor of evaluations but also facilitates the comparison of 
  different SDGFTS models, ultimately providing a more standardized and reliable means of assessing 
  synthetic data quality.
  \subsection{Summary of Findings}
  \subsection{Implications for SDGFTS Research}
  \subsection{Current Limitations and Shortcomings}
  \subsection{Future Research Directions}


\begin{acks}
To professor Irene Huang, University of Toronto, for her supervision and guidance throughout the project.
\end{acks}


\bibliographystyle{ACM-Reference-Format}
\bibliography{stonk-bench}



\appendix

\section{Preprocessing Diagnostics}
  \subsection{Channel-wise Distributions}
  \begin{figure*}[h]
    \centering
    % \includegraphics[width=0.95\textwidth]{results/evaluation_YYYYMMDD_HHMMSS/GBM/distributions.png}
    \caption{Histograms and Q--Q plots of real vs. synthetic returns per channel. Replace with generated figures from \texttt{VisualAssessmentEvaluator}.}
  \end{figure*}

  \subsection{Autocorrelation Analyses}
  \begin{figure*}[h]
    \centering
    % \includegraphics[width=0.95\textwidth]{results/evaluation_YYYYMMDD_HHMMSS/GBM/acf_pacf.png}
    \caption{ACF/PACF of returns and absolute returns for real vs. synthetic data.}
  \end{figure*}

\section{Embedding Visualizations}
  \begin{figure*}[h]
    \centering
    % \includegraphics[width=0.95\textwidth]{results/evaluation_YYYYMMDD_HHMMSS/GBM/tsne.png}
    \caption{t-SNE/UMAP embeddings of window-level features; color-coded real vs. synthetic for overlap assessment.}
  \end{figure*}

\section{Per-Model Visual Assessments}
  For each model (GBM, OU\_Process, MJD, GARCH11, DEJD, BlockBootstrap, TimeGAN, QuantGAN, TimeVAE), include side-by-side real/synthetic plots and distribution overlays.
  \begin{figure*}[h]
    \centering
    % Example placeholder for one model; duplicate per model and update path
    % \includegraphics[width=0.95\textwidth]{results/evaluation_YYYYMMDD_HHMMSS/GBM/overview.png}
    \caption{Representative visual assessment panels for a selected model.}
  \end{figure*}

\end{document}